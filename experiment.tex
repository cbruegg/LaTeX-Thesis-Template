\chapter{Experiment}
\label{chap:experiment}
In diesem Kapitel werden Aufbau und Durchführung des Experiments erläutert, durch das der Datensatz zusammengestellt wurde.

Insgesamt haben \todo{n} Teilnehmer am Experiment teilgenommen und die in Abbildung~\ref{fig:activities} achtzehn gelisteten Aktivitäten durchgeführt. Weiss et al. nahmen jeweils zwei Minuten pro Teilnehmer und Aktivität auf\cite{Weiss2016}, mussten allerdings zum Anfang und Ende jeder Aufnahme je zehn Sekunden abschneiden, um Ausreißer zu entfernen. Aus diesem Grund betrug die Dauer jeder Aufnahme in diesem Experiment drei Minuten, wodurch der gesamte Aufnahmeprozess pro Person rund zwei Stunden dauerte. Vor dem Experiment wurden Einverständniserklärungen der Teilnehmer eingeholt.

Die Aufnahme erfolgte mit dem Fitness-Tracker Microsoft Band 2 und dem Smartphone OnePlus 3, auf dem das Betriebssystem Android 6 installiert war. Der Fitness-Tracker wurde am Handgelenk des Teilnehmers befestigt, während das Smartphone in seiner Hosentasche platziert wurde, wie in \cite{Weiss2016} jeweils auf der dominanten Seite des Teilnehmers. Dies war notwendig, da insbesondere Aktivitäten wie das Dribbeln eines Basketballs mit der dominanten Hand durchgeführt werden. Während dies für Armbanduhren nicht notwendigerweise eine realistische Konfiguration ist, da diese üblicherweise auf der nicht-dominanten Seite getragen werden \todo{cite}, besteht dieses Problem bei Fitness-Trackern nicht unbedingt. So lässt sich beispielsweise in Fitness-Trackern der Firma Fitbit einstellen, auf welcher Seite dieser getragen wird, was die Vermutung nahelegt, dass genügend Nutzer mit dem Tragen auf dieser Seite kein Problem haben oder dies sogar bevorzugen. \todo{cite https://help.fitbit.com/articles/en\_US/Help\_article/1136}

Auf dem Smartphone wurde eine eigens für das Experiment entwickelte Anwendung ausgeführt, in der zunächst der Name des Teilnehmers eingegeben wurde. Sowohl auf dem Smartphone selbst als auch per drahtloser Fernsteuerung wurde die durchzuführende Aktivität eingestellt. Gestartet und gestoppt wurde die Aufnahme anschließend per Fernsteuerung, um Ausreißer am Anfang und Ende der Aufnahme zu vermeiden, obgleich trotzdem jeweils zehn Sekunden abgeschnitten wurden.

\begin{figure}
    \smaller 
    \centering
    \begin{itemize}
        \item Allgemeine Aktivitäten (nicht handorientiert)
        \begin{itemize} \item Gehen \end{itemize}
        \begin{itemize} \item Joggen \end{itemize}
        \begin{itemize} \item Treppensteigen \end{itemize}
        \begin{itemize} \item Sitzen \end{itemize}
        \begin{itemize} \item Stehen \end{itemize}
        \begin{itemize} \item Fußball schießen \end{itemize}
        \item Allgemeine Aktivitäten (handorientiert)
        \begin{itemize} \item Basketball dribbeln \end{itemize}
        \begin{itemize} \item Mit einem Tennisball Fangen spielen \end{itemize}
        \begin{itemize} \item Auf einer Tastatur tippen \end{itemize}
        \begin{itemize} \item Auf Papier schreiben \end{itemize}
        \begin{itemize} \item Klatschen \end{itemize}
        \begin{itemize} \item Zähneputzen \end{itemize}
        \begin{itemize} \item Kleidung falten \end{itemize}
        \item Essaktivitäten (handorientiert)
        \begin{itemize} \item Spaghetti essen \end{itemize}
        \begin{itemize} \item Suppe essen \end{itemize}
        \begin{itemize} \item Brot essen \end{itemize}
        \begin{itemize} \item Chips essen \end{itemize}
        \begin{itemize} \item Aus einer Tasse oder einem Glas trinken \end{itemize}
    \end{itemize}
    \caption{Aktivitäten des Experiments nach \cite{Weiss2016}}
\end{figure}
\label{fig:activities}

\section{Beschreibung der Aktivitäten}
Im Folgenden werden die einzelnen Aktivitäten, die von den Teilnehmern ausgeführt wurden, genauer beschrieben.

\subsection{Allgemeine Aktivitäten (nicht handorientiert)}
\subsubsection{Gehen}
Der Teilnehmer bewegt sich im Schrittempo auf einem Gehweg.
\subsubsection{Joggen}
Der Teilnehmer joggt auf einem Gehweg. Das Tempo variiert je nach sportlicher Verfassung des Teilnehmers.
\subsubsection{Treppensteigen}
Der Teilnehmer bewegt sich abwechselnd eine Treppe hoch unter herunter. Steigung und Länge sind variabel.
\subsubsection{Sitzen}
Der Teilnehmer sitzt auf einem Stuhl und versucht, sich dabei nicht unruhig zu verhalten.
\subsubsection{Stehen}
Der Teilnehmer steht und versucht, sich dabei nicht unruhig zu verhalten.
\subsubsection{Fußball schießen}
Der Teilnehmer schießt einen Fußball wiederholt mit mittlerer Kraft zu einem Mitspieler und versucht, Sprinten zu vermeiden.

\subsection{Allgemeine Aktivitäten (handorientiert)}
\subsubsection{Basketball dribbeln}
Der Teilnehmer schleudert einen Basketball wiederholt Richtung Boden und versucht, dabei möglichst an einer Stelle stehen zu bleiben.
\subsubsection{Mit einem Tennisball Fangen spielen}
Zwei Personen werfen sich abwechselnd einen Tennisball zu und versuchen dabei, möglichst an einer Stelle stehen zu bleiben.
\subsubsection{Auf einer Tastatur tippen}
Der Teilnehmer tippt sitzend seinen Gewohnheiten nach einen Text an einer beliebigen Computertastatur ab. Mindestens grobe Fehler sollten korrigiert werden. Um Verständnisproblemen aus dem Weg zu gehen, handelt es sich bei dem Text um ein Diktat für Siebtklässler.
\subsubsection{Auf Papier schreiben}
Der Teilnehmer schreibt sitzend seinen Gewohnheiten nach denselben Text wie mit der Tastatur mit einem Kugelschreiber auf ein Blatt Papier im Format DIN A4 ab.
\subsubsection{Klatschen}
Der Teilnehmer begleitet sitzend das Lied "Viva la Vida" von Coldplay klatschend.
\subsubsection{Zähneputzen}
Der Teilnehmer benutzt stehend eine Handzahnbürste (nicht elektrisch), um sich damit die Zähne zu putzen.
\subsubsection{Kleidung falten}
Der Teilnehmer faltet stehend der Reihe nach T-Shirts auf einem Tisch.

\subsection{Essaktivitäten (handorientiert)}
\subsubsection{Spaghetti essen}
Der Teilnehmer isst Spaghetti mit einer Soße.
\subsubsection{Suppe essen}
Der Teilnehmer isst eine Suppe seiner Wahl.
\subsubsection{Brot essen}
Der Teilnehmer isst ein belegtes Brot.
\subsubsection{Chips essen}
Der Teilnehmer isst Chips aus einer handelsüblichen Tüte.
\subsubsection{Aus einer Tasse oder einem Glas trinken}
Der Teilnehmer trinkt wiederholt aus einer Tasse oder einem Glas.