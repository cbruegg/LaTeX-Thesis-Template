\chapter{Experiment}
\label{chap:experiment}
In diesem Kapitel werden Aufbau und Durchführung des Experiments erläutert, durch das der Datensatz zusammengestellt wurde.

Insgesamt haben 10 Teilnehmer am Experiment teilgenommen und die in Abbildung~\ref{sec:activities} achtzehn gelisteten Aktivitäten durchgeführt. Weiss et al. nahmen jeweils zwei Minuten pro Teilnehmer und Aktivität auf \cite{Weiss2016}, mussten allerdings zum Anfang und Ende jeder Aufnahme je zehn Sekunden abschneiden, um Ausreißer zu entfernen. Aus diesem Grund betrug die Dauer jeder Aufnahme in diesem Experiment drei Minuten, wodurch der gesamte Aufnahmeprozess pro Person rund zwei Stunden dauerte. Vor dem Experiment wurden Einverständniserklärungen der Teilnehmer eingeholt.

Die Aufnahme erfolgte mit dem Fitness-Tracker Microsoft Band 2 und dem Smartphone OnePlus 3, auf dem das Betriebssystem Android 6 installiert war. Der Fitness-Tracker wurde am Handgelenk des Teilnehmers befestigt, während das Smartphone in in einer frontseitigen Hosentasche platziert wurde, wie in \cite{Weiss2016} jeweils auf der dominanten Seite des Teilnehmers. Dies war notwendig, da insbesondere Aktivitäten wie das Dribbeln eines Basketballs mit der dominanten Hand durchgeführt werden. Die Ausrichtung des Smartphones in der Hosentasche selbst war im Experiment nicht festgelegt.

Auf dem Smartphone wurde eine eigens für das Experiment entwickelte Anwendung ausgeführt, in der zunächst der Name des Teilnehmers eingegeben wurde. Sowohl auf dem Smartphone selbst als auch per drahtloser Fernsteuerung wurde die durchzuführende Aktivität eingestellt. Gestartet und gestoppt wurde die Aufnahme anschließend per Fernsteuerung, um Ausreißer am Anfang und Ende der Aufnahme zu vermeiden, obgleich sicherheitshalber trotzdem jeweils zehn Sekunden abgeschnitten wurden.

\section{Beschreibung der Aktivitäten}
\label{sec:activities}
Im Folgenden werden die einzelnen Aktivitäten, die von den Teilnehmern ausgeführt wurden, genauer beschrieben. Um eine Vergleichbarkeit mit \cite{Weiss2016} zu ermöglichen, handelt es sich mangels detaillierter Beschreibungen zumindest um ähnliche Aktivitäten, die dieselben Bezeichnungen haben.

\subsection*{Allgemeine Aktivitäten (nicht handorientiert)}
\subsubsection{Gehen}
Der Teilnehmer bewegt sich im Schrittempo auf einem Gehweg.
\subsubsection{Joggen}
Der Teilnehmer joggt auf einem Gehweg. Das Tempo variiert je nach sportlicher Verfassung des Teilnehmers.
\subsubsection{Treppensteigen}
Der Teilnehmer bewegt sich abwechselnd eine Treppe hoch und herunter. Steigung und Länge sind variabel.
\subsubsection{Sitzen}
Der Teilnehmer sitzt auf einem Stuhl und versucht, sich dabei nicht unruhig zu verhalten.
\subsubsection{Stehen}
Der Teilnehmer steht und versucht, sich dabei nicht unruhig zu verhalten.
\subsubsection{Fußball schießen}
Der Teilnehmer schießt einen Fußball wiederholt zu einem wenige Meter entfernten Mitspieler und versucht, Sprints zu vermeiden.

\subsection*{Allgemeine Aktivitäten (handorientiert)}
\subsubsection{Basketball dribbeln}
Der Teilnehmer wirft einen Basketball wiederholt auf den Boden und versucht, dabei möglichst an einer Stelle stehen zu bleiben.
\subsubsection{Mit einem Tennisball Fangen spielen}
Zwei Personen werfen sich abwechselnd einen Tennisball zu und versuchen dabei, möglichst an einer Stelle stehen zu bleiben.
\subsubsection{Auf einer Tastatur tippen}
Der Teilnehmer tippt sitzend seinen Gewohnheiten nach einen Text an einer beliebigen Computertastatur ab. Mindestens grobe Fehler sollten korrigiert werden. Um Leseschwierigkeiten aus dem Weg zu gehen, handelt es sich bei dem Text um ein Diktat für Siebtklässler.
\subsubsection{Auf Papier schreiben}
Der Teilnehmer schreibt sitzend seinen Gewohnheiten nach denselben Text wie mit der Tastatur mit einem Kugelschreiber auf ein Blatt Papier im Format DIN A4 ab.
\subsubsection{Klatschen}
Der Teilnehmer klatscht im Takt sitzend zum Lied "Viva la Vida" von Coldplay.
\subsubsection{Zähneputzen}
Der Teilnehmer putzt sich stehend mit einer nicht-elektrischen Handzahnbürste die Zähne.
\subsubsection{Kleidung falten}
Der Teilnehmer faltet stehend der Reihe nach Kleidung auf einem Tisch.

\subsection*{Essaktivitäten (handorientiert)}
\subsubsection{Spaghetti essen}
Der Teilnehmer isst Spaghetti mit einer beliebigen Soße.
\subsubsection{Suppe essen}
Der Teilnehmer isst eine beliebige Suppe.
\subsubsection{Brot essen}
Der Teilnehmer isst ein belegtes Brot.
\subsubsection{Chips essen}
Der Teilnehmer isst Chips aus einer handelsüblichen Tüte.
\subsubsection{Aus einer Tasse oder einem Glas trinken}
Der Teilnehmer trinkt wiederholt aus einer Tasse oder einem Glas.

\section{Überwachung der Aufnahme}
Zur Qualitätssicherung der Daten wurde die Aufnahme der Aktivitäten überwacht. Den Teilnehmern wurde größtmögliche Freiheit bei der Ausführung gegeben, weshalb lediglich auf starke Abweichungen von der Aufgabenstellung hingewiesen wurde, die über einen Zeitraum von über 10 Sekunden hinweg begangen wurden. Als Beispiele seien hierfür das beabsichtigte Werfen des Tennisballs auf den Boden und das Gestikulieren in der Luft während des Tippens auf der Tastatur genannt, wobei das eigentliche Tippen auf dieser eingestellt wurde.

Um eine sonstige Beeinflussung der Bewegungsabläufe zu vermeiden, wurde auf einen festen Aufnahmeort in einem Labor verzichtet, so wie es auch Bao \& Intille zumindest für einige Aktivitäten vermieden haben \cite{Bao2004}. Stattdessen wurden die Experimente in einem freizeitlichen Kontext und in unterschiedlichen Umgebungen durchgeführt, sodass sich die Teilnehmer möglichst wie in ihrem sonstigen Alltag verhalten würden. In keinem Fall wurde eine Aufnahme, die schon über 10 Sekunden lief, aufgrund von Fehlverhalten abgebrochen oder nicht weiterverwendet.

\section{Auswahl der Teilnehmer}
\label{sec:users}
Bei der Auswahl der Teilnehmer wurde darauf geachtet, dass die Gruppe nicht übermäßig homogen wurde. Wie Tabelle~\ref{tab:user-attributes} zeigt, reicht die Altersspanne zum Zeitpunkt des Experiments von 16 bis 50, wobei der maximale Körpergrößenunterschied 24 cm beträgt. Unter den 10 Teilnehmern sind drei weiblich und die verschiedenen Berufsstände zeigen die unterschiedlichen Milieuzugehörigkeiten der Personen. Des Weiteren ist einer der 10 Teilnehmer Linkshänder und trug die Geräte somit auf seiner linken Seite. Dieser Anteil entspricht schätzungsweise dem Anteil von Linkshändern unter den seit 1940 geborenen Personen \cite{mcmanus2010science}. Teilnehmer 2 ist der Autor dieser Bachelorarbeit.
\begin{figure}
\centering
\begin{tabular}{|c||c|c|c|c|c|}
	\hline 
	\textbf{Teilnehmer} & \textbf{Geschlecht} & \textbf{Alter} & \textbf{Beruf} & \textbf{Größe} & \textbf{Dom. Seite} \\ 
	\hline 
	1 & w & 47 & Zahnmed. Fachangestellte & 171 cm & R \\ 
	\hline 
	2 & m & 22 & Student & 176 cm & R \\ 
	\hline 
	3 & m & 20 & Konstruktionsmechaniker & 187 cm & R \\ 
	\hline 
	4 & m & 16 & Schüler & 182 cm & R \\ 
	\hline 
	5 & w & 18 & Med. Fachangestellte & 163 cm & R \\ 
	\hline 
	6 & m & 21 & Student & 181 cm & R \\ 
	\hline 
	7 & m & 50 & Ingenieur & 183 cm & R \\ 
	\hline 
	8 & m & 22 & Student & 183 cm & L \\ 
	\hline 
	9 & w & 19 & Studentin & 163 cm & R \\ 
	\hline 
	10 & m & 50 & Kraftfahrer & 178 cm & R \\ 
	\hline 
\end{tabular} 
\caption{Attribute der Teilnehmer}
\label{tab:user-attributes}
\end{figure}


Da nun ein Datensatz zum Training und zur Evaluation besteht, kann im folgenden Kapitel mit der Entwicklung der Methode fortgefahren werden.