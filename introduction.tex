\chapter{Einleitung}
\label{chap:introduction}
\section{Motivation}
Smartphones haben im letzten Jahrzehnt an großer Bedeutung gewonnen. Daneben existieren mittlerweile ergänzend dazu sogenannte \textit{Smartwatches} und \textit{Fitness-Tracker}. Smartwatches sind Armbanduhren, die in der Regel drahtlos mit einem Smartphone verbunden sind und Informationen wie beispielsweise Benachrichtigungen am Handgelenk zugänglich machen. Fitness-Tracker besitzen ähnliche Funktionen, zielen allerdings primär darauf ab, die Fitness und Gesundheit des Nutzers zu fördern, indem Daten wie beispielsweise die Schrittanzahl des Nutzers pro Tag gesammelt und grafisch aufbereitet werden. In beiden Geräteformen werden üblicherweise Sensoren verbaut, mit denen sich die Bewegungen des Trägers nachvollziehen lassen.

Mit einigen Fitness-Trackern des Unternehmens \textit{Fitbit} existieren bereits kommerzielle Produkte, die über die reine Sammlung und grafische Aufbereitung von Daten hinausgehen: Die Funktion \textit{SmartTrack} erkennt kontinuierliche Aktivitäten mit hoher Bewegung mit Hilfe von Sensordaten des Trackers teilweise automatisch, ohne dass der Anwender vorher manuell einstellen muss, welcher Aktivität er in den nächsten Minuten nachgehen wird \cite{FitbitSmartTrack}. Dies hat den Vorteil, dass der Nutzer sich nicht daran erinnern muss, im Fitness-Tracker die richtige Aktivität einzustellen, um kategorisierte Statistiken zu erhalten.

SmartTrack unterstützt die folgenden Aktivitäten: Gehen, Laufen, Fahrradfahren, Schwimmen und Training mit einem Crosstrainer, sowie zwei allgemeine Kategorien "Sport" (Fußball, Basketball, Tennis, etc.) und "aerobes Training" (Zumba, Tanzen).

Es existieren weitere mögliche Anwendungsgebiete der automatisierten Aktivitätenerkennung. Für Smartphone-Betriebssysteme könnte das Wissen, dass der Anwender gerade Sport treibt, interessant sein, um eingehende Anrufe eines nicht als wichtig markierten Kontaktes zu unterdrücken. Des Weiteren könnte das Forschungsgebiet der "Transportation Mode Recognition" von solchen Methoden profitieren: Soll erkannt werden, mit welchem Verkehrsmittel sich der Nutzer gerade fortbewegt, könnte neben dem Parameter der Geschwindigkeit ebenfalls von Interesse ein, ob mit Hilfe der Methode die Aktivität "Fahrradfahren" erkannt wird oder nicht. So ließe sich die Fortbewegung mittels eines Mofas von der Fortbewegung mittels eines Fahrrads unterscheiden, was insbesondere für Dienste wie "Google Now" nützlich sein könnte. Diese dienen dem Nutzer als persönlicher Assistent und warnen ihn beispielsweise vor Stau auf einer häufig befahrenen Strecke. Eine solche Warnung könnte entfallen, wenn festgestellt wurde, dass der Nutzer die Strecke nicht mit einem Motorroller, sondern mit einem Fahrrad bewältigt und somit Radwege befahren darf.

In der Literatur ist \cite{Weiss2016} hervorzuheben. Die Autoren vergleichen in ihrem Paper die Genauigkeit der Aktivitätenerkennung eines Smartphones mit der einer Smartwatch und kommen zu dem Schluss, dass die Güte der jeweiligen Erkennung insbesondere von der Aktivität selbst abhängig ist. Es liegt auf der Hand, dass nur mit Hilfe eines Smartphones beispielsweise eine Unterscheidung zwischen "Zähneputzen" und "Stehen" schwer möglich ist, während analog dazu nur mit Hilfe einer Smartwatch die Unterscheidung zwischen "Gehen" und "Fußball schießen" ebenfalls herausfordernd ist. Naheliegend ist daher, eine Kombination beider Datenquellen einzusetzen, um die durchschnittliche Erkennungsrate zu verbessern, ohne eine Beschränkung der erkennbaren Aktivitäten einzuführen.

\section{Ziele der Arbeit} % Problemdefinition
\label{sec:goals}
Evaluiert werden soll ein zu entwickelndes Verfahren, das mit Methoden des \textit{maschinellen Lernens} (siehe Definition~\ref{def:ml-classification}) und eben jenen gesammelten Daten feststellt, welcher Aktivität der Träger der Geräte in bestimmten Zeitintervallen nachgegangen ist. Hierzu wird zunächst eine Software benötigt, welche die synchrone Aufzeichnung von Sensordaten eines Smartphones und zusätzlich eines Fitness-Trackers oder einer Smartwatch ermöglicht.

Es ergibt sich insbesondere die Frage, inwiefern sowohl personalisierte, das heißt nutzerspezifische, als auch unpersonalisierte Modelle durch die Hinzunahme einer weiteren Datenquelle genauer werden.

Um eine Evaluation zu ermöglichen, wird ein Beispieldatensatz benötigt, der durch ein Experiment mit 10 Probanden aufgebaut wird. Orientiert ist diese Zahl an der Anzahl der Probanden in \cite{Weiss2016}, an dessen Experiment 17 Personen teilgenommen haben. Im Experiment sollen diese voneinander unabhängig mehreren definierten Aktivitäten nachgehen, während parallel dazu Sensordaten mithilfe der entwickelten Software aufgezeichnet werden. Um die Vergleichbarkeit mit den Ergebnissen aus \cite{Weiss2016} zu gewährleisten, werden die Probanden in diesem Experiment denselben Aktivitäten nachgehen.

Als Fitness-Tracker wurde das \textit{Microsoft Band 2} ausgewählt, da dieser vom Lehrstuhl für diese Arbeit zur Verfügung gestellt wurde, im Vergleich zur privat vorhandenen Smartwatch \textit{Pebble Time} mehr Sensoren besitzt und letztere während der Durchführung des Experiments nach einem erzwungenen Software-Update falsche Zeitstempel für Sensordaten lieferte. 

\section{Wichtige Ergebnisse}
Hervorzuheben sind die Erkenntnisse, dass die Genauigkeit persönlicher Modelle mithilfe der Kombination der in Abschnitt~\ref{sec:sensors} vorgestellten Sensoren auf $99.4 \%$ gesteigert werden konnte. Im Vergleich zu Modellen, die nur auf Daten des Beschleunigungssensors eines Fitness-Trackers basieren, beträgt die Steigerung damit $7.8$ Prozentpunkte. Unpersönliche Modelle hingegen konnten durch diese Technik um $3.3$ Prozentpunkte auf $78.5 \%$ verbessert werden, wobei gleichzeitig die Stabilität insofern gesteigert wurde, dass die schlechteste Genauigkeit für einen Teilnehmer des Experiments mit einem unpersönlichen Modell nicht unter $64 \%$ lag. Um die Genauigkeit unpersönlicher Modelle weiter zu verbessern, empfiehlt es sich, nur klar voneinander abgrenzbare Aktivitäten zu klassifizieren und beispielsweise mehrere Essaktivitäten zu vermeiden.

Des Weiteren konnte festgestellt werden, dass auch niedrige, einstellige Sensorabtastraten in Hz bei einem niedrigeren Energieverbrauch noch gute Ergebnisse liefern können und die Variation hinsichtlich der Ausrichtung der Geräte nur einen geringen Einfluss auf die Genauigkeiten der Modelle hat.

\section{Struktur dieser Arbeit}
Kapitel~\ref{chap:background} erläutert die verwendeten Sensoren sowie wichtige Begriffe des \ac{ML} als Grundlagen dieser Arbeit. Im darauffolgenden Kapitel~\ref{chap:relatedwork} werden verwandte Arbeiten aufgezählt und historisch eingeordnet, da insbesondere die Entwicklung mobiler Technologie diverse Fortschritte im Bereich der Aktivitätenerkennung erst ermöglicht hat. Anschließend daran wird in Kapitel~\ref{chap:experiment} der Aufbau des Experiments beschrieben, das im Rahmen dieser Bachelorarbeit durchgeführt wurde, um die benötigten Daten zu sammeln. Die Verarbeitung der gewonnenen Daten wird in Kapitel~\ref{chap:method} erläutert. Die aus der Anwendung der Methode resultierenden Modelle werden in Kapitel~\ref{chap:evaluation} evaluiert, wobei unter anderem diverse Eigenschaften des aufgenommenen Datensatzes variiert werden, um Erkenntnisse über den Wert der Sensoren für die Aktivitätenerkennung zu erhalten. Ein abschließendes Fazit der Arbeit erfolgt in Kapitel~\ref{chap:conclusions}.

% vim: set ft=tex
