\chapter{Grundlagen}
\label{chap:background}
\section{Erläuterung der verwendeten Sensoren}
\todo{Grafiken zur Erläuterung}
Während in der Einleitung nur generisch von "Sensordaten" die Rede war, werden diese nun konkretisiert. Zur Aufnahme werden das Smartphone OnePlus 3 sowie der Fitness-Tracker Microsoft Band 2 verwendet. Beide Geräte besitzen einen Beschleunigungssensor sowie ein Gyroskop. Des Weiteren besitzt das Band unter anderem einen Sensor, der den elektrischen Widerstand der Haut des Trägers misst, sowie einen Hauttemperatursensor.

\subsection{Beschleunigungssensor}
Ein Beschleunigungssensor, auch \textit{Accelerometer} genannt, misst die echte Beschleunigung eines Objektes im Raum je physikalischer Achse ($x$, $y$ und $z$). Dies beinhaltet auch die Beschleunigung, die von der Erdgravitation ausgeht: Liegt das Objekt beispielsweise auf dem Boden, erfährt es auf der im rechten Winkel zum Boden stehenden Achse eine Beschleunigung von $g \approx 9.81$ $m/s^2$\cite{SensorsOverview}\cite{nistsi}. Über diesen Sensor kann demnach die Orientierung des Objektes im Raum bestimmt werden. \\
Die Daten dieses Sensors werden von beiden Geräten aufgezeichnet.

\subsection{Gyroskop}
Ein Gyroskop misst die Rotationsgeschwindigkeit je physikalischer Achse\cite{SensorsOverview}. Demnach bedeutet der Messwert $(x=0, y=0, z=0)$, dass das Objekt relativ zur Erde nicht bewegt wird, während $(x=1, y=0, z=0)$ bedeutet, dass das Objekt um die $x$-Achse gedreht wird. \\
Die Daten dieses Sensors werden von beiden Geräten aufgezeichnet.

\subsection{Hautwiderstandssensor}
Je nach aktueller körperlicher Betätigung ändert sich die elektrische Leitfähigkeit der Hautoberfläche durch Schweiß. Der Hautwiderstandssensor misst den elektrischen Widerstand der Haut, der sich umgekehrt proportional zur Leitfähigkeit verhält.
Dieser Sensor ist nur im Band integriert, das daher für diesen die einzige Datenquelle ist.

\subsection{Hauttemperatursensor}
Der Hauttemperatursensor misst die Temperatur der Haut des Nutzers, die an der Unterseite des Band anliegt. Leider reagierte dieser Sensor in einem Test nur langsam auf Temperaturveränderungen: Nachdem der Tracker abgelegt wurde, übermittelte dieser den Temperaturabfall erst mehrere Sekunden später. Aus diesem Grund wurde auf die Aufzeichnung dieser Daten verzichtet.

\subsection{Laufgeschwindigkeitssensor}
Der Laufgeschwindigkeitssensor ist ein virtueller Sensor, der auf Basis des elektronischen Schrittzählers im Microsoft Band 2 generiert wird. Er gibt die Laufgeschwindigkeit des Trägers in $cm/s$ an.

\section{Wichtige Begriffe}
In den folgenden Kapiteln wird von einem grundlegenden Verständnis von Machine Learning-Klassifikation ausgegangen.

\begin{definition}[ML-Klassifikation]
\label{def:ml-classification}
Gegeben sei ein Datensatz $D \ni (x_1, ..., x_n, y)$ mit $|D| = m$ Instanzen. Dann sind $x = (x_1, ..., x_n)$ die \textit{Input-Features} und $y$ das \textit{Target}. Im Falle eines Klassifikationsproblems ist $y$ nominal, d.h. ein deskriptiver Wert zur Identifikation. Gesucht ist nun eine \textit{Hypothesenfunktion} $h(x) = \hat{y}$, die zu gegebenen Features den wahrscheinlichsten Wert des Targets bestimmt\cite{Ng2011a}.
\end{definition}
Neben Definition~\ref{def:ml-classification} sind für das Verständnis dieser Arbeit die folgenden Begriffe wichtig:
\begin{enumerate}
\item (Überwachtes) Lernen, bzw. Training. Definiert in Kapitel 2.1 \cite{Hastie2009}.
\item Overfitting und Train-Test-Split. Definiert und erläutert in Kapitel 7.2 \cite{Hastie2009}.
\item Kreuzvalidierung. Definiert und erläutert in Kapitel 7.10 \cite{Hastie2009}.
\end{enumerate}


% vim: set ft=tex
