\chapter{Grundlagen}
\label{chap:background}
\section{Erläuterung der verwendeten Sensoren}
\todo{Grafiken zur Erläuterung}
Während in der Einleitung nur generisch von "Sensordaten" die Rede war, werden diese nun konkretisiert. Zur Aufnahme werden das Smartphone OnePlus 3 sowie der Fitness-Tracker Microsoft Band 2 verwendet. Beide Geräte besitzen einen Beschleunigungssensor sowie ein Gyroskop. Des Weiteren besitzt das Band unter anderem einen Sensor, der den elektrischen Widerstand der Haut des Trägers misst, sowie einen Hauttemperatursensor.

\subsection{Beschleunigungssensor}
Ein Beschleunigungssensor, auch \textit{Accelerometer} genannt, misst die echte Beschleunigung eines Objektes im Raum je physikalischer Achse ($x$, $y$ und $z$). Dies beinhaltet auch die Beschleunigung, die von der Erdgravitation ausgeht: Liegt das Objekt beispielsweise auf dem Boden, erfährt es auf der im rechten Winkel zum Boden stehenden Achse eine Beschleunigung von $g \approx 9.81$ $m/s^2$\cite{SensorsOverview}\cite{nistsi}. Über diesen Sensor kann demnach die Orientierung des Objektes im Raum bestimmt werden. \\
Die Daten dieses Sensors werden von beiden Geräten aufgezeichnet.

\subsection{Gyroskop}
Ein Gyroskop misst die Rotationsgeschwindigkeit je physikalischer Achse\cite{SensorsOverview}. Demnach bedeutet der Messwert $(x=0, y=0, z=0)$, dass das Objekt relativ zur Erde nicht bewegt wird, während $(x=1, y=0, z=0)$ bedeutet, dass das Objekt um die $x$-Achse gedreht wird. \\
Die Daten dieses Sensors werden von beiden Geräten aufgezeichnet.

\subsection{Hautwiderstandssensor}
Je nach aktueller körperlicher Betätigung ändert sich die elektrische Leitfähigkeit der Hautoberfläche durch Schweiß. Der Hautwiderstandssensor misst den elektrischen Widerstand der Haut, der sich umgekehrt proportional zur Leitfähigkeit verhält.
Dieser Sensor ist nur im Band integriert, das daher für diesen die einzige Datenquelle ist.

\subsection{Hauttemperatursensor}
Der Hauttemperatursensor misst die Temperatur der Haut des Nutzers, die an der Unterseite des Band anliegt. Leider reagierte dieser Sensor in einem Test nur langsam auf Temperaturveränderungen: Nachdem der Tracker abgelegt wurde, übermittelte dieser den Temperaturabfall erst mehrere Sekunden später. Aus diesem Grund wurde auf die Aufzeichnung dieser Daten verzichtet.

\section{Grundlagen der Machine Learning-Klassifikation}
In den folgenden Abschnitten wird von einem grundlegenden Verständnis von Machine Learning-Klassifikation ausgegangen, weshalb wir diese Technik nun kurz erläutern. \\
\begin{definition}[ML-Klassifikation]
Gegeben sei ein Datensatz $D \ni (x_1, ..., x_n, y)$ mit $|D| = m$ Instanzen. Dann sind $x = (x_1, ..., x_n)$ die \textit{Input Features} und $y$ ist das \textit{Target}. Im Falle eines Klassifikationsproblems ist $y$ nominal, d.h. ein deskriptiver Wert zur Identifikation. Gesucht ist nun eine \textit{Hypothesenfunktion} $h(x) = \hat{y}$, die zu gegebenen Features den wahrscheinlichsten Wert des Targets bestimmt\cite{Ng2011a}.
\end{definition}\label{def:ml-classification}

Ein einfaches Beispiel kann wie folgt konstruiert werden: Sei $x = (x_1)$, wobei $x_1$ die Wohnfläche einer Wohnung oder eines Hauses in $m^2$ ist. Des Weiteren sei $D = \{(50, \text{Wohnung}), (75, \text{Wohnung}), (200, \text{Haus}), (300, \text{Haus})\}$. Eine sinnvolle Hypothesenfunktion wäre in diesem Fall beispielsweise gegeben durch:

\[
h(x) = 
\begin{cases}
\text{Wohnung}, & \text{wenn } x_1 < 200 \\
\text{Haus}, & \text{sonst}
\end{cases}
\]

Das beschriebene Klassifikationsproblem kann mittels Algorithmen des \textit{Machine Learnings} (im Folgenden \textit{ML}) gelöst werden, indem eine Verlustfunktion definiert wird, die ein solcher Algorithmus zu minimieren versucht. Eine solche Verlustfunktion bestraft die fehlerhafte Klassifikation einer Instanz, indem sie einen höheren Wert annimmt. Die Ausführung des ML-Algorithmus wird auch \textit{Lernen} genannt.

Wenn ein ML-Algorithmus auf einem Datensatz arbeitet, besteht die Gefahr, dass er diesen gewissermaßen "auswendig lernt", d.h. sich diesem zu sehr anpasst und auf unbekannten Daten schlechte Vorhersagen liefert. Dieses Problem wird \textit{Overfitting} genannt und kann erkannt werden, indem der Datensatz $D$ in zwei disjunkte Mengen $D = \text{Train} \uplus \text{Test}$ aufgeteilt wird (\textit{Train-Test-Split}). Der ML-Algorithmus wird nun auf den Trainingsdaten ausgeführt, woraufhin dessen Ergebnis mittels den Testdaten auf Overfitting geprüft wird.

Ein übliches Verfahren ist dabei die $k$-Kreuzvalidierung. Mit $D = D_1 \uplus ... \uplus D_k$ wird der ML-Algorithmus $k$ mal ausgeführt, wobei in Iteration $i$ gilt, dass $\text{Train} = D \setminus D_i$ und $\text{Test} = D_i$. Effektiv bedeutet dies, dass dem Algorithmus in jeder Iteration ein anderer Teil der Daten vorenthalten wird. Das Ergebnis der Kreuzvalidierung ist die Hypothese, die auf den jeweiligen Testdaten das beste Ergebnis liefert.

% vim: set ft=tex
