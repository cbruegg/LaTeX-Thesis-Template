\chapter*{Zusammenfassung}
Handelsübliche Smartphones und Smartwatches sowie Fitness-Tracker enthalten Sensoren, mit denen sich die körperlichen Aktivitäten des Benutzers aufzeichnen und auswerten lassen. Frühere Forschung widmete sich bereits der Aktivitätenerkennung mittels Smartphones und Smartwatches, jedoch wurden die Daten bisher voneinander getrennt behandelt. Diese Bachelorarbeit widmet sich der Kombination der Datenquellen, um die Vorteile beider Gerätetypen zu vereinen und Modelle mittels maschinellem Lernen zu bilden, die sowohl handorientierte als auch allgemeine Aktivitäten mit hoher Genauigkeit klassifizieren können. Die entwickelte Methode wurde anhand eines Datensatzes getestet, der im Rahmen dieser Arbeit durch ein Experiment mit 10 Personen aufgenommen wurde.

% vim: set ft=tex
