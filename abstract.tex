\chapter*{Zusammenfassung}
Handelsübliche Smartphones und Smartwatches sowie Fitness-Tracker enthalten Sensoren, mit denen sich die körperlichen Aktivitäten des Benutzers aufzeichnen und auswerten lassen. Frühere Forschung widmete sich bereits der Aktivitätenerkennung mittels Smartphones und Smartwatches, jedoch wurden die Sensordaten bisher voneinander getrennt behandelt. Diese Bachelorarbeit widmet sich der Kombination der Datenquellen, um Modelle mittels maschinellem Lernen zu bilden, die sowohl handorientierte als auch allgemeine Aktivitäten mit hoher Genauigkeit klassifizieren können. Die entwickelte Methode wurde anhand eines Datensatzes getestet, der im Rahmen dieser Arbeit durch eine Datenerhebung mit 10 Personen aufgenommen wurde. Die Genauigkeit persönlicher Modelle konnte um 7.8 Prozentpunkte auf $99.4 \%$ gesteigert werden, während die Genauigkeit unpersönlicher Modelle um 3.3 Prozentpunkte auf $78.5 \%$ verbessert werden konnte. Des Weiteren konnte im Zuge der Prüfung diverser Einflüsse auf den Datensatz unter anderem festgestellt werden, dass auch energiesparsame Sensorabtastraten im einstelligen Hz-Bereich noch gute Ergebnise liefern können und die Variation hinsichtlich der Ausrichtung der Geräte nur einen geringen Einfluss auf die Genauigkeiten hat.

% vim: set ft=tex
