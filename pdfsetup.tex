%%% Set up hyperref if we are compiling with pdflatex
%%
\ifx\pdftexversion\undefined
\usepackage{hyperref}
\else
\usepackage[colorlinks=false,       %true => text des links wird farbig, false => farbiges kaestchen (wird nicht gedruckt) um schwarzen text
            linkcolor=black,        %nur bei colorlinks=true
            urlcolor=black,
            citecolor=black,
            bookmarks,                  % Bookmarks erstellen
            bookmarksopen=true,         % Bookmark beim Oeffnen anzeigen?
            pdfpagemode=UseOutlines,    % Bookmark beim Oeffnenanzeigen? (UseOutlines / none)
            bookmarksopenlevel=3,       % bis zu welcher Ebenen geoeffnet
            bookmarksnumbered,          % Kapitelnummern in Bookmarks
            pdftitle={Erkennung körperlicher Aktivitäten mittels Smartphone- und Smartwatch-Sensoren und Machine Learning},
            pdfsubject={Bachelorarbeit},
            pdfkeywords={Smartphone, Smartwatch, Bachelorarbeit, Machine Learning, Aktivitätenerkennung},
            pdfauthor={Christian Brüggemann}
            ]{hyperref}
\fi

%%% 
%%
\ifx\pdftexversion\undefined
\else
\pdfminorversion=6
\pdfoutput=1        % PDF-Ausgabe anschalten.
\pdfimageresolution=600
\pdfcompresslevel=5 % 0 keine kompression, 9 staerkste kompression
\fi

% vim: set ft=tex
