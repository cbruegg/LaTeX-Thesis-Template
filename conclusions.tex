\chapter{Fazit}
\label{chap:conclusions}
\todo{Check template for suggestions on how to write all of this}
\todo{Irgendwo unterbringen: Intervalle: Man könnte mehrere Intervallgrößen ausprobieren, aber dadurch könnte sich Overfitting ergeben}

\section{Zusammenfassung}
\todo{TODO}

\section{Beurteilung der Ergebnisse}
\todo{TODO}

\section{Zusammenhang zum Kontext}
\todo{TODO}

\section{Ausblick}
\todo{TODO}
\todo{Mehrere Sensoren durch Korrelationsfeatures noch besser miteinander kombinieren. }
\todo{Features wie Zero-Cross (Dernbach 2012), Richtungsänderungen im Graphen}
\todo{Anschließend Relevanz der Features prüfen}
\todo{Veröffentlichung des Datensatzes folgt auf eigener Webseite.}
\todo{Inkrementeller Aufbau hybrider Modelle durch Frage an User: Wurde korrekt klassifiziert?. Irgendwann wird laut Weiss hybrides Modell schlechter als persönliches, dann umsteigen, sofern alle Aktivitäten aufgezeichnet wurden. [Ist das Reinforcement Learning?]}
\todo{Features mehrerer aufeinanderfolgender Intervalle, um Sequentialität zu modellieren. Curse of Dimensionality}

\section{Reflexion}
\todo{TODO}
\todo{Daten des Accelerometers/Gyroskops? hinsichtlich Orientierung normalisieren, wobei ein guter ML-Algorithmus dies vielleicht auch selbst schaffen kann. Magnetometer war am Anfang irrelevant, dafür wäre es aber nützlich gewesen.}


% vim: set ft=tex
